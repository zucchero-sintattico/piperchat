\section{Analisi dei Requirements}

Di seguito vengono formalizzati i requisiti:

\subsection{Funzionali}

\begin{enumerate}
    \item \textbf{Registrazione e Autenticazione}:
    \begin{enumerate}[label*=\arabic*.]
        \item Possibilità di registrazione al sistema.
        
        \item Possibilità di login nel sistema.
    \end{enumerate}
    \item \textbf{Sistema di amicizie}:
    \begin{enumerate}[label*=\arabic*.]
        \item Possibilità di inviare richieste di amicizia ad altri utenti.
        
        \item Possibilità di accettare o rifiutare richieste di amicizia.
        
        \item Sistema di notifiche per la ricezione di nuovo richieste di amicizia.
        
        \item Sistema di gestione dello stato (online) e dell'ultimo accesso degli utenti.
    \end{enumerate}
    \item \textbf{Interazioni con amici}:
    \begin{enumerate}[label*=\arabic*.]
        \item Gestione messaggistica tra i propri amici.
        
        \item Sistema di notifiche per l'arrivo di nuovi messaggi.

        \item Possibilità di entrare in sessione con un amico.
    \end{enumerate}
    \item \textbf{Gestione Server}:
    \begin{enumerate}[label*=\arabic*.]
        \item Possibilità di creazione di un nuovo server.
        
        \item Possibilità di entrare un server esistente.
        
        \item Possibilità da parte del creatore del server di rimuovere i membri.
    \end{enumerate}
    \item \textbf{Gestione canali}:
    \begin{enumerate}[label*=\arabic*.]
        \item Possibilità di creazione di canali (testuali o multimediali) da parte del creatore del server.
    
        \item Possibilità di rimozione di canali da parte del creatore del server.

        \item Sistema di messaggistica per i canali testuali.

        \item Sistema di notifiche per i nuovi messaggi inviati all'interno di canali testuali.

    % \end{enumerate}
    % \item \textbf{Gestione canali multimediali}:
    % \begin{enumerate}[label*=\arabic*.]
        % \item Possibilità di creazione di canali multimediali da parte del creatore del server.
        % \item Possibilità di rimozione di canali multimediali da parte del creatore del server.
        \item Possibilità di accedere alla sessione dei canali multimediali.
    \end{enumerate}

    \item \textbf{Gestione sessione}:
    \begin{enumerate}[label*=\arabic*.]
        \item Possibilità di comunicare attraverso gli altri partecipanti alla stessa sessione.

        \item Possibilità di accendere e spegnere il microfono e la fotocamera.
    \end{enumerate}

    \item \textbf{Sistema di monitoring dei microservizi}:
    \begin{enumerate}[label*=\arabic*.]
        \item Sistema di monitoraggio dello stato dei servizi.
    \end{enumerate}
    
\end{enumerate}

%
%
%
\subsection{Non funzionali}

\begin{enumerate}
    \setcounter{enumi}{7}

    \item \textbf{Sicurezza}:
    \begin{enumerate}[label*=\arabic*.]
        \item Autenticazione degli utenti per verificarne l'identità.

        \item Autorizzazione degli utenti per l'accesso alle risorse in base alle regole stabilite.

        \item Crittografia per garantire la confidenzialità delle password degli utenti.
    \end{enumerate}
    
    \item \textbf{Scalabilità}:
    \begin{enumerate}[label*=\arabic*.]
        \item Scalabilità orizzontale per gestire l'aumento del carico operativo.
    \end{enumerate}

    \item \textbf{Manutenibilità}:
    \begin{enumerate}[label*=\arabic*.]
        \item Modularità degli artefatti.
        \item Codice sorgente ben strutturato e comprensibile.
    \end{enumerate}

\end{enumerate}