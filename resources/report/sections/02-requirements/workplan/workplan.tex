\section{Expected work plan}

Is there any implicit requirement hidden within this project's requirements?
%
Is there any implicit hypothesis hidden within this project's requirements?
%
Are there any non-functional requirements implied by this project's requirements?

What model / paradigm / techonology is the best suited to face this project's requirements?
%
What's the abstraction gap among the available models / paradigms / techonologies and the problem to be solved?

Lo sviluppo del progetto sarà guidato dalla seguente scaletta, in modo da rendere l'integrazione delle features incrementali.

\begin{enumerate}

    \item Sviluppo scheletro infrastruttura a micro-servizi.
    \begin{enumerate}
        \item Selezione e integrazione del Broker.
        \item Sviluppo sistema Logging per il monitoraggio degli eventi del sistema (utile per monitorare l'incremento delle features)
        \item Sviluppo dashboard reattiva per la visualizzazione dello stato dei servizi.
    \end{enumerate}
    
    \item Sviluppo sistema di utenti.
    \begin{enumerate}
        \item Sistema di registrazione.
        \item Sistema di login
        \item Sistema di gestione stato (online, offline, ultimo accesso, etc..)
    \end{enumerate}

    \item Sviluppo sistema di notifiche
    \begin{enumerate}
        \item Sistema di collegamento tramite websockets.
        \item Sistema di memorizzazione di ricezione delle notifiche.
    \end{enumerate}

    \item Sviluppo sistema di messaggistica intra-utenti
    \begin{enumerate}
        \item Sistema di richieste di amicizia.
        \item Aggiunta supporto notifiche richieste amicizia.
        \item Sistema chat private tra utenti.
        \item Aggiunta supporto notifiche messaggi.
    \end{enumerate}

    \item Sviluppo sistema di gestione dei server.
    \begin{enumerate}
        \item Sistema di creazione server.
        \item Sistema di join dei server.
        \item Sistema di creazione canali testuali all'interno dei server.
    \end{enumerate}

    \item Estensione della messaggistica all'audio/video.
    \begin{enumerate}
        \item Sistema di signaling WebRTC per inizializzazione chiamate.
        \item Sistema di creazione di canali multimediali.
        \item Sistema di chiamate multimediali intra-utenti.
        \item Aggiunta supporto notifiche chiamate in entrata.
    \end{enumerate}

    \item Feature opzionali
    \begin{enumerate}
        \item Sistema di blacklisting intra-utenti.
        \item Sistema di ban degli utenti dai server.
        \item Sviluppo di un servizio adibito al salvataggio dei file inviati dagli utenti in modo da abilitare non solo messaggi testuali ma anche l’invio di file.
        \item Gestione auto-scaling.
    \end{enumerate}

\end{enumerate}

\subsection{Technologies}

Di seguito le tecnologie che verranno utilizzate per l'implementazione del progetto:

\begin{itemize}
    \item Node.js per lo sviluppo dei servizi
    \begin{itemize}
        \item Typescript come linguaggio
        \item Express.js per il server
        \item Socket.io per la comunicazione tramite Websocket
        \item Mongoose per la gestione del database
        \item Amqplib per la connessione al broker
        \item Jest per il testing
    \end{itemize}
    \item MongoDB per i database dei servizi
    \item RabbitMQ come broker
    \item Docker + Docker-Compose per l’infrastruttura
    \item HTTP e Websocket come protocolli di comunicazione con i servizi
    \item HTML, Javascript e CSS per il client Web.
    \item WebRTC per la comunicazione audio/video
\end{itemize}
