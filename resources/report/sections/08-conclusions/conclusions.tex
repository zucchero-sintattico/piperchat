\chapter{Conclusioni}

Il progetto "Piperchat" ha portato alla creazione di una piattaforma di comunicazione ispirata a Discord, con l'obiettivo di offrire un ambiente dove gli utenti possono interagire in varie forme.
%
Durante il processo di sviluppo, sono stati identificati diversi requisiti funzionali, tra cui la registrazione e l'autenticazione degli utenti, la gestione delle amicizie, la messaggistica intra-utente e la gestione dei server e dei canali.
%
Inoltre, sono stati definiti requisiti non funzionali per garantire usabilità e accessibilità.
%
Il design del progetto è stato basato su un modello iterativo e orientato all'utente, con una particolare attenzione all'usabilità e all'accessibilità.
%
Sono stati creati mockup per valutare diverse combinazioni cromatiche e configurazioni dell'interfaccia utente, garantendo un'esperienza visiva ottimale.
%
L'architettura del sistema è stata realizzata utilizzando un'architettura a microservizi, suddividendo la complessità in parti più piccole e coese.
%
Sono state utilizzate diverse tecnologie, MEVN (MongoDB, Express, Vue.js, Node.js), Docker per la gestione dei contenitori, Swagger per la documentazione delle API, Traefik per il routing del traffico, WebRTC per le videochiamate, Socket.IO per la comunicazione in tempo reale e altre librerie e framework.
%
Infine, è stata creata un'organizzazione chiara delle API, definendo le specifiche di ogni endpoint e utilizzando TypeScript per garantire un typing accurato.
%
In conclusione, il progetto "Piperchat" ha portato alla creazione di una piattaforma di comunicazione completa, con attenzione all'usabilità, all'accessibilità e alle esigenze specifiche degli utenti.
%
La sua architettura a microservizi e l'organizzazione delle API lo rendono scalabile e flessibile per future espansioni e miglioramenti.
